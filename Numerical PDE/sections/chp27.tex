\chapter{Lax-Wendroff method and upwind methods}
One way to achieve second order accuracy on the advection equation is to use a second order temporal discretization of the system of ODEs \eqref{eq: MOL for advect equation LS} since this system is based on a second order spatial discretization. This can be done with the midpoint method, for example, which gives rise to the leapfrog scheme already discussed. However, this is a three-level method and for various reasons it is often much more convenient to use two-level methods for PDEs whenever possible-in more than one dimension the need to store several levels of data may be restrictive, boundary conditions can be harder to impose, and combining methods using fractional step procedures may require two-level methods for each step, to name a few reasons. Moreover, the leapfrog method is nondissipative, leading to potential stability problems if the method is extended to variable coefficient or nonlinear problems.

Another way to achieve second order accuracy in time would be to use the trapezoidal method to discretize the system, as was done to derive the Crank-Nicolson method for the heat equation. But this is an implicit method and for hyperbolic equations there is generally no need to introduce this complication and expense.

Another possibility is to use a two-stage Runge-Kutta method. This can be done, although some care must be exercised near boundaries, and the use of a multistage method again typically requires additional storage.

\section{The Lax-Wendroff method}
 
One simple way to achieve a two-level explicit method with higher accuracy is to use the idea of Taylor series methods. Applying this directly to the linear system of ODEs $U^{\prime}(t)=A U(t)$ (and using $U^{\prime \prime}=A U^{\prime}=A^2 U$ ) gives the second order method
\begin{align*}
U^{n+1}=U^n+k A U^n+\frac{1}{2} k^2 A^2 U^n .
\end{align*}

Here $A$ is the matrix\eqref{eq: A of MOL advect eq LS} and computing $A^2$ and writing the method at the typical grid point then gives
\begin{align*}
U_j^{n+1}=U_j^n-\frac{a k}{2 h}\left(U_{j+1}^n-U_{j-1}^n\right)+\frac{a^2 k^2}{8 h^2}\left(U_{j-2}^n-2 U_j^n+U_{j+2}^n\right) .
\end{align*}
This method is second order accurate and explicit but has a 5-point stencil involving the points $U_{j-2}^n$ and $U_{j+2}^n$. With periodic boundary conditions this is not a problem, but with other boundary conditions this method needs more numerical boundary conditions than a 3-point method. This makes it less convenient to use and potentially more prone to numerical instability.

Note that the last term is an approximation to $\frac{1}{2} a^2 k^2 u_{x x}$ using a centered difference based on step size $2 h$. A simple way to achieve a second order accurate 3-point method is to replace this term by the more standard 3-point formula. We then obtain the standard \textbf{Lax-Wendroff method}:
\begin{align}
    \label{eq: Lax-Wendroff method}
U_j^{n+1}=U_j^n-\frac{a k}{2 h}\left(U_{j+1}^n-U_{j-1}^n\right)+\frac{a^2 k^2}{2 h^2}\left(U_{j-1}^n-2 U_j^n+U_{j+1}^n\right) .
\end{align}

A cleaner way to derive this method is to use Taylor series expansions directly on the $\operatorname{PDE} u_t+a u_x=0$, to obtain
\begin{align*}
u(x, t+k)=u(x, t)+k u_t(x, t)+\frac{1}{2} k^2 u_{t t}(x, t)+\cdots .
\end{align*}
Replacing $u_t$ by $-a u_x$ and $u_{t t}$ by $a^2 u_{x x}$ gives
\begin{align*}
u(x, t+k)=u(x, t)-k a u_x(x, t)+\frac{1}{2} k^2 a^2 u_{x x}(x, t)+\cdots .
\end{align*}
If we now use the standard centered approximations to $u_x$ and $u_{x x}$ and drop the higher order terms, we obtain the Lax-Wendroff method. It is also clear how we could obtain higher order accurate explicit two-level methods by this same approach, by retaining more terms in the series and approximating the spatial derivatives (including the higher order spatial derivatives that will then arise) by suitably high order accurate finite difference approximations. The same approach can also be used with other PDEs. The key is to replace the time derivatives arising in the Taylor series expansion with spatial derivatives, using expressions obtained by differentiating the original PDE.

We can analyze the stability of Lax-Wendroff following the same approach used for LaxFriedrichs. Note that with periodic boundary conditions, the Lax-Wendroff method can be viewed as Euler's method applied to the linear system of ODEs $U^{\prime}(t)=A_\epsilon U(t)$, where $A_\epsilon$ is given by \eqref{eq: A of MOL advect eq LS} with $\epsilon=a^2 k / 2$ (instead of the value $\epsilon=h^2 / 2 k$ used in Lax-Friedrichs). The eigenvalues of $A_\epsilon$ are
\begin{align*}
k \mu_p=-i\left(\frac{a k}{h}\right) \sin (p \pi h)+\left(\frac{a k}{h}\right)^2(\cos (p \pi h)-1) .
\end{align*}

These values all lie on an ellipse centered at $-(a k / h)^2$ with semi-axes of length $(a k / h)^2$ and $|a k / h|$. If $|a k / h| \leq 1$, then all of these values lie inside the stability region of Euler's method. Figure~\ref{fig 10.1} shows an example in the case $a k / h=0.8$. The Lax-Wendroff method is stable with exactly the same time step restriction as required for LaxFriedrichs. Later we will see that this is a very natural stability condition to expect for the advection equation and is the best we could hope for when a 3-point method is used.

A close look at Figure~\ref{fig 10.1} shows that the values $k \mu_p$ near the origin lie much closer to the boundary of the stability region for the Lax-Wendroff method (Figure~\ref{fig 10.1}) than for the other methods illustrated in this figure. This is a reflection of the fact that LaxWendroff is second order accurate, while the others are only first order accurate. Note that a value $k \mu_p$ lying inside the stability region indicates that this eigenmode will be damped as the wave propagates, which is unphysical behavior since the true solution advects with no dissipation. For small values of $\mu_p$ (low wave numbers, smooth components) the LaxWendroff method has relatively little damping and the method is more accurate. Higher wave numbers are still damped with Lax-Wendroff (unless $|a k / h|=1$, in which case all the $k \mu_p$ lie on the boundary of $\mathcal{S}$ ) and resolving the behavior of these modes properly would require a finer grid.

\section{Upwind methods} 
So far we have considered methods based on symmetric approximations to derivatives. Alternatively, one might use a nonsymmetric approximation to $u_x$ in the advection equation, e.g.,
\begin{align*}
u_x\left(x_j, t\right) \approx \frac{1}{h}\left(U_j-U_{j-1}\right)
\end{align*}
or
\begin{align*}
u_x\left(x_j, t\right) \approx \frac{1}{h}\left(U_{j+1}-U_j\right)
\end{align*}
These are both one-sided approximations, since they use data only to one side or the other of the point $x_j$. Coupling one of these approximations with forward differencing in time gives the following methods for the advection equation:
\begin{align}
    \label{eq: upwind}
U_j^{n+1}=U_j^n-\frac{a k}{h}\left(U_j^n-U_{j-1}^n\right)
\end{align}
or
\begin{align}
    \label{eq: downwind}
U_j^{n+1}=U_j^n-\frac{a k}{h}\left(U_{j+1}^n-U_j^n\right) .
\end{align}
These methods are first order accurate in both space and time. One might wonder why we would want to use such approximations, since centered approximations are more accurate. 

For the advection equation, however, there is an asymmetry in the equations because the equation models translation at speed $a$. If $a>0$, then the solution moves to the right, while if $a<0$ it moves to the left. There are situations where it is best to acknowledge this asymmetry and use one-sided differences in the appropriate direction.

The choice between the two methods \eqref{eq: upwind} and \eqref{eq: downwind} should be dictated by the sign of $a$. Note that the true solution over one time step can be written as
\begin{align*}
u\left(x_j, t+k\right)=u\left(x_j-a k, t\right)
\end{align*}
so that the solution at the point $x_j$ at the next time level is given by data to the left of $x_j$ if $a>0$, whereas it is determined by data to the right of $x_j$ if $a<0$. This suggests that \eqref{eq: upwind} might be a better choice for $a>0$ and \eqref{eq: downwind} for $a<0$.
In fact the stability analysis below shows that \eqref{eq: upwind} is stable only if
\begin{align*}
0 \leq \frac{a k}{h} \leq 1
\end{align*}
Since $k$ and $h$ are positive, we see that this method can be used only if $a>0$. This method is called the \textbf{upwind} method when used on the advection equation with $a>0$. If we view the equation as modeling the concentration of some tracer in air blowing past us at speed $a$, then we are looking in the correct upwind direction to judge how the concentration will change with time. (This is also referred to as an upstream differencing method in some literature.)
Conversely, \eqref{eq: downwind} is stable only if
\begin{align*}
-1 \leq \frac{a k}{h} \leq 0
\end{align*}
and can be used only if $a<0$. In this case \eqref{eq: downwind} is the proper upwind method to use.

The method \eqref{eq: upwind} can be written as
\begin{align*}
U_j^{n+1}=U_j^n-\frac{a k}{2 h}\left(U_{j+1}^n-U_{j-1}^n\right)+\frac{a k}{2 h}\left(U_{j+1}^n-2 U_j^n+U_{j-1}^n\right),
\end{align*}
which puts it in the form \eqref{eq: Aeps LF advect eq} with $\epsilon=a h / 2$. We have seen previously that methods of this form are stable provided $|a k / h| \leq 1$ and also $-2<-2 \epsilon k / h^2<0$. Since $k, h>0$, this requires in particular that $\epsilon>0$. For Lax-Friedrichs and Lax-Wendroff, this condition was always satisfied, but for upwind the value of $\epsilon$ depends on $a$ and we see that $\epsilon>0$ only if $a>0$. If $a<0$, then the eigenvalues of the MOL matrix lie on a circle that lies entirely in the right half-plane, and the method will certainly be unstable. If $a>0$, then the above requirements lead to the stability restriction.

The three methods, Lax-Wendroff, upwind, and Lax-Friedrichs, can all be written in the same form \eqref{eq: Aeps LF advect eq} with different values of $\epsilon$. If we call these values $\epsilon_{L W}, \epsilon_{u p}$, and $\epsilon_{L F}$, respectively, then we have
\begin{align*}
\epsilon_{L W}=\frac{a^2 k}{2}=\frac{a h v}{2}, \quad \epsilon_{u p}=\frac{a h}{2}, \quad \epsilon_{L F}=\frac{h^2}{2 k}=\frac{a h}{2 v},
\end{align*}
where $v=a k / h$. Note that
\begin{align*}
\epsilon_{L W}=v \epsilon_{u p} \quad \text { and } \quad \epsilon_{u p}=v \epsilon_{L F}
\end{align*}
If $0<v<1$, then $\epsilon_{L W}<\epsilon_{u p}<\epsilon_{L F}$ and the method is stable for any value of $\epsilon$ between $\epsilon_{L W}$ and $\epsilon_{L F}$. 

\subsection{The Beam-Warming method}
The upwind method is only first order accurate. A second order accurate method with the same one-sided character can be derived by following the derivation of the Lax-Wendroff method, but using one-sided approximations to the spatial derivatives. This results in the \textbf{Beam-Warming} method, which for $a>0$ takes the form
\begin{align*}
U_j^{n+1}=U_j^n-\frac{a k}{2 h}\left(3 U_j^n-4 U_{j-1}^n+U_{j-2}^n\right)+\frac{a^2 k^2}{2 h^2}\left(U_j^n-2 U_{j-1}^n+U_{j-2}^n\right) .
\end{align*}
For $a<0$ the Beam-Warming method is one-sided in the other direction:
\begin{align*}
U_j^{n+1}=U_j^n-\frac{a k}{2 h}\left(-3 U_j^n+4 U_{j+1}^n-U_{j+2}^n\right)+\frac{a^2 k^2}{2 h^2}\left(U_j^n-2 U_{j+1}^n+U_{j+2}^n\right) .
\end{align*}
These methods are stable for $0 \leq v \leq 2$ and $-2 \leq v \leq 0$, respectively.