\part{The Initial Value Problem}
\chapter{IVP for ODE}
In this chapter we begin a study of time-dependent differential equations, beginning with the initial value problem (IVP) for a time-dependent ordinary differential equation (ODE).

%────────────────────────────────────────
\begin{problem}
[Initial value problem for ODE]
\label{prob: Initial value problem}
The IVP takes the form 
\begin{equation}
\label{eq: IVP}
    \begin{cases}
        u'(t) = f(u(t),t) & \forall t>t_0 \\
        u(t_0) = \eta  & 
    \end{cases}
\end{equation}
For convenience, we often assume $ t_0=0 $. Here we call $ \eta  $ the initial condition and $ u(t) $ the state. 
\end{problem}
%────────────────────────────────────────
Note that here $ u:[0,T]\to \RR^d $ is a vector function. We will use the following notations: 
\begin{notation}
    \begin{itemize}
        \item []
        \item Newton dot notation: $ \frac{dx}{dt} (t) = \dot x (t) $,
        \item $ p $-th order derivative: $ \frac{d^{p}x}{dt^{p}}(t) = x^{(p)}(t) $. 
    \end{itemize}
\end{notation}
 

\section{Reduction to autonomous case} 

%────────────────────────────────────────
\begin{definition}
[Autonomous]
\label{def: Autonomous}
An first order ODE is autonomous if 
\[
    u'(t) = f(u(t)).
\]
In other words, $ f $ doesn't depend directly on $ t $. Otherwise, we call the ODE non-autonomous.  
\end{definition}
%────────────────────────────────────────
 
We have the following proposition:

%────────────────────────────────────────
\begin{proposition}
\label{prop: Reduction to autonomous}
Any IVP problem \eqref{eq: IVP} can be reduced to the autonomous case.
\end{proposition}
%────────────────────────────────────────
%────────────────────────────────────────
\begin{proof}[Proof Sketch]
We can do this by adding one dimension to the state variable. We let $ v(t) = (u(t)^\top , t)^\top  $. Then we have 
\[
    \begin{cases}
        v'(t)  = \begin{pmatrix}[] 
             g(v(t)) \\
             1 \\
        \end{pmatrix}  \quad \forall t>t_0 \\
        v(t_0) = \begin{pmatrix}[] 
             \eta  \\
             t_0 \\
        \end{pmatrix}  
    \end{cases}
\]
where $ g(v(t)) = f(u(t), t) $. This ODE is autonomous. 
\end{proof}
%────────────────────────────────────────

\section{Reduction to first-order case} 
A more general ODE is of the form: 
\[
    u^{(p)}(t) = F(u(t), u^\prime (t), u^{\prime \prime}(t), \ldots ,u^{(p-1)}(t), t).  
\] 


%────────────────────────────────────────
\begin{proposition}
\label{prop: Reudction to first-order case}
A more general ODE can be reduced to \eqref{eq: IVP}. 
\end{proposition}
%────────────────────────────────────────
%────────────────────────────────────────
\begin{proof}[Proof sketch]
Just consider  
\[
    v(t) = \begin{bmatrix}[] 
         u(t) \\
         u^\prime (t) \\
         \vdots \\
         u^{(p-1)}(t) \\
    \end{bmatrix}.  
\]
\end{proof}
%────────────────────────────────────────

\section{Linear ODEs} 

%────────────────────────────────────────
\begin{definition}
[Linear ODEs]
\label{def: Linear ODEs}
The system \eqref{eq: IVP} is linear if 
\begin{equation}
\label{eq: linear ODE}
    f(u,t) = A(t)u + g(t),
\end{equation}
where $ A(t) \in \RR^{d\times d}, g(t) \in \RR^{d} $. A special case is the constant coefficient linear system
\begin{equation}
\label{eq: constant linear ODE}
    f(u,t) = Au(t) + g(t), 
\end{equation}
where $ A $ is a constant matrix. If $ g(t)\equiv 0 $, the equation is homogeneous. 
\end{definition}
%────────────────────────────────────────
 
The solution to a homogeneous system $ u' = Au $ is 
\[
    u(t) = e^{A(t-t_0)}\eta . 
\]

We have the Duhamel's principle for linear ODEs. 

%────────────────────────────────────────
\begin{theorem}
[Duhamel's principle]
\label{thm: Duhamel's principle}
The solution to \eqref{eq: constant linear ODE} can be written as 
\begin{equation}
\label{eq: Duhamel's principle}
    u(t) = e^{A(t-t_0)}\eta  + \int_{t_0}^{t} e^{A(t-\tau )}g(\tau)\, d\tau. 
\end{equation}
Especially, if $ g(t)\equiv g$, then this can be reduced to 
\begin{equation}
\label{eq: Duhamel's principle g constant}
    u(t) = e^{A(t-t_0)}\eta  + A^{-1}  (e^{A(t-t_0)}-I) g. 
\end{equation}

\end{theorem}
%────────────────────────────────────────
