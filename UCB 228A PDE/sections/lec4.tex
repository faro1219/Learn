\newpage 
\section{First order nonlinear Scalar PDEs III}
\textbf{Date:} Sep 7
Given two ODEs:
\[
  \begin{cases}
      u' = F(t, u)\\
      u(0) = u_0 
  \end{cases}  
  \text{ and } 
  \begin{cases}
      v' = F(t,v)\\
      v(0) = v_0
  \end{cases}
\]
we are going to show that: 

\begin{theorem}
    Suppose that the solution $u$ exists on $[0,T]$. Then there exists $\varepsilon>0$ such that if $|v_0-u_0|\le \varepsilon$, then $v$ exists on $[0,T]$ and $\|u-v\|_c \le c|u_0-v_0|$. In other words, the map $u_0 \to u_{[0,T]}$ is locally Lipschitz.
\end{theorem}

\begin{proof}
    We compute that 
    \[
        \frac{d}{dt}|u-v|^2 = 2(u-v) \cdot \frac{d}{dt}(u-v) = 2(u-v) (F(u) - F(v)) \le 2|u-v| \cdot L |u-v| = 2L|u-v|^2.
    \]  
    Hence with Gronwall's inequality, 
    \[
        f(t) \le f(0) e^{2Lt}.  
    \]
    The proof is finished except that we don't know that $V$ exists up to full $[0,T]$. Let 
    \begin{align*}
        D_1 = \{v\in \con[0,T], |v-u|\le 1\} \\
        D_2 = \{v\in \con[0,T], |v-u|\le 2\}      
    \end{align*}
    Suppose we know $v\in D_2$, then $v$ is defined on $[0,T]$and satay in a compact set, so the above argument applies.

    Suppose this is not true, we use $T_2$ to denote the time the function exists from $D_2$. Then we can use Gronwall up to $T_2$. 
    \[
    |u(t)-v(t)|^2 \le |u_0-v_0|^2 \cdot e^{2LT_2} \le \varepsilon^2 e^{2LT}.  
    \]
    When we set $\varepsilon$ small enough, we have $|u(t) -v(t)|\le 1 , \forall t\in [0,T_2]$. Hence $v$ doesn't exist $D_1$. Contradiction.
\end{proof}

\begin{remark}
    Here we also use Bootstrap assumption. The key ideal is that we use a weaker assumption to prove a stronger results. Here we use $|u-v|\le 2$ to get $|u-v|\le 1$. 
\end{remark}

Now we assume $F\in \con^1$ and $u_0^h$ takes a one parameter family of data. When $h$ close to $0$, $u_0^h$ is differentiable in $h$. From data $u_0^0$, we get $u^0$ and from $u_0^h$, we get $u^h$.

\text{Question:} How does $u_h$ depend on $h$? We already know that 
\[
    |u_0^h -u_0^0| \le h \Rightarrow |u^h - u^0|\le he^{2Lt}.  
\]
But we can give a more formal computation for $v^h = \frac{d}{dh}u^h$. We can get a equation
\[
    \dot{v}^h = D F(t,u^h) v^h, \quad v^h(0) = \frac{d}{dh}u^h_0.
\]
This equation is called the Linearized equation. This is because: 
\[
    \frac{d}{dt}(u^h-u^0) = F(t, u^h(t)) - F(t, u^0(t)) = DF(t, u^0(t))u^h(t) - u^0(t)) + o(u^h(t) - u^0(t))^2.
\]

\subsection{First order scalar PDE}
The general form of first order scalar PDE is: 
\[
    u:\RR^n \to \RR \quad  F(x,u, \partial u) = 0.    
\]
We can classify them by degree of difficulty: linear, semilinear, quasilinear and fully nonlinear PDEs. THe initial data is given on a surface. Besides, it can also be divided into static PDEs and dynamic PDEs. 

\subsection{ Linear problem}

The general form of linear problem is: 
\[
    \sum A_j(x) \partial_j u = bu + f.    
\]
We need curves which are tangent to $A=(A_1(x),...,A_n(x))$ at each point. Hence we are going to solve two ODEs: 
\[
    \begin{cases}
        \dot{x}(t) = A(x(t))\\ 
        \frac{d}{dt} u(x(t)) = \nabla u x(t) = A \cdot \nabla(t) = b u (x(t) ) + f.
    \end{cases}    
\]