\newpage 
\section{First order nonlinear scalar PDEs I}
\textbf{Date:} Aug 31, 2021

In the following three weeks, we are going to consider 
\[
    F(x, \mu, \partial \mu) = 0.    
\]

Firstly, we are going to consider its functional spaces:
\begin{itemize}
    \item Which function could be a solution?
    \item How do we verify that a function is a solution?
\end{itemize}

We define some common functions spaces first:

\begin{definition}
[$\con$, $\con^m$, $\con^m_{loc}$ and $\con^\infty$ functions] Given $\RR^n$,
\begin{itemize}
    \item $\con(\RR^n)$ is the set of bounded continuous functions, and $\con_{loc}(\RR^n)$ is the set of continuous function with $\con(\RR^n) \subset \con_{loc}(\RR^n)$. For $\mu\in \con(\RR^n)$, the norm is defined as
    \[
            \|\mu\|_{\con} = \sup_{x\in \RR^n} |\mu(x)|. 
    \]
    \item $\con^m(\RR^n)$ is the set of bounded functions with $m$-order derivatives. For any derivative with order smaller than $m$, the derivative is also bounded.  $\con^m_{loc}(\RR^n)$ is the set of functions only $m$-order differentiable. The norm of $\con^m(\RR^n)$ is defined as:
    \[
        \|u\|_{\con^m} = \sum_{i=0}^n \|D^i u\|_{\con}.    
    \]
    \item $\con^\infty(\RR^n) = \cap \con_{loc}^m (\RR^n)$.
\end{itemize}
\end{definition}
For $\con_{loc}$, we cannot define norms but we can consider a series of interval $I_N= [-N, N]$, and 
\[
    \|\mu\|_{C[I_N]} = \sup_{x\in I_N} |\mu(x)|.    
\]
We say $\mu_n \to \mu$ in $\con_{loc}$, if $\|\mu_n -\mu\|_{C[I_N]}$ converges to $0$ for all $N$. We call space with such a series of norms the locally convex space, which is the extension of norm space.

Now we review some knowledge in ODE. Given ODE,
\[
\mu : \RR \to \RR^n, \begin{cases}
    \mu'(x) = F(x, \mu(x))\\
    \mu(0) = \mu_0
\end{cases}    
\]

We may ask about? 
\begin{itemize}
    \item The existence of solution
    \item The uniqueness of solution 
    \item This depends on the initial data
    \item Local vs global equations
\end{itemize}

\subsection{Local Solution}
At a minimum, we may ask that $F$ is continuous and look for a $\con^1$ local solution $h$.

\begin{theorem}
    [Peano] If $f$ continuous, then a local $\con^1$ solution exists.
\end{theorem}

\begin{example}
    Solution of $u'(x) = \sqrt{u}, u(0)=0$ is not unique. For any $c>0$,
    \[
    f(x) = 
    \begin{cases}
        0 & x\le c\\
        (x-c)^2/4 &x>c
    \end{cases}    
    \]
    is a solution.
\end{example}

\begin{definition}
    [Lipschitz continuous function] Give a function $f$ from a Banach space to $\RR$, we call it Lipschitz, if 
    \[
        |F(x) - F(y)| \le L |x-y|, \quad \forall x,y    
    \]
    where $L$ is a constant in $\RR$.
\end{definition}
\begin{remark}
    $\con^1 \subset \Lip $ but $\Lip$ is not a subset of $\con^1$. An example is $f(x) = |x|$.
\end{remark}

\begin{definition}
    [Holder continuous function]
    We call a function $f$ is holder continuous if
    \[
        |F(x) - F(y)| \le M |x-y|^s,    
    \] 
    for some constant $M$ and $0<s<1$.
\end{definition}

\begin{theorem}
    If $f$ is locally Lipschitz, then a local solution exists and is unique.
\end{theorem}

\begin{theorem}
    Given $f:\RR \to \RR$ and $|f'| <1$, f has an unique fixed point. 
\end{theorem}

\begin{theorem}
    Given a Lipschitz function $f: B \to B$, with constant $<1$, $f$ has a unique fixed point. 
\end{theorem}

\begin{theorem}
    Given a Lipschitz function $f:D\subset B\to D$ where $D$ is a closed set and the Lipschitz
\end{theorem}