\newpage
\section{Existence of Solutions to Nonlinear First Order Scalar PDEs}
\textbf{Date:} Sep 14, 2021

\subsection{Existence and uniqueness given initial data}
Last time, we were looking at fully nonlinear equations
\[
\begin{cases}
    F(x, u, \partial u) = 0\\
    u = u_0 \text{ on }\Sigma. 
\end{cases}
\]
If $u$ solves this equation and $x$ solve the 1.(a) ODE, then $(x,u,\partial_j u)$ solves the characteristic system 
\[
    \left\{\begin{array}{l}
        \dot{x}=F_{p}(x, z, p) \\
        \dot{z}=F_{p}(x, z, p) \cdot p \\
        \dot{p}=-F_{x}(x, z, p)-F_{z}(x, z, p) \cdot p
        \end{array}\right.
\]
The initial data for the characteristic system on $\Sigma$ is 
\[
    \left\{\begin{array}{l}
        x(0)=x_{0} \\
        z(0)=u_{0}\left(x_{0}\right) \\
        p(0)=p_{0}
        \end{array}\right.
\]
where $p_0$ has a tangential component $\partial_\tau u_0$ and a normal component given by solving $F(x_0, u_0, p_0)$. In this last part, we had a local solvability condition $F_p\cdot N \neq 0$, where $N$ is the normal to $\Sigma$. This is the same as the noncharacteristic condition.  Our objective is to turn this into an existence proof. 

\begin{theorem}
    Assume that $F$ is of clsss $\con^2$, $\Sigma$ is $\con^2$, $\mu_0\in \con^2$, and the problem is noncharacteristic, i.e., there exists $p_0$ on $\Sigma$ such that $F_{p_0}\cdot N \neq 0, F(x_0, u_0, p_0)=0$, and $(p_0)_\tau = \partial_\tau u_0$. Then there exists a unique local solution $u\in \con^2$ near $\Sigma$ such that $u|_\Sigma= u_0$ and $\partial u|_\Sigma = p_0$.
\end{theorem}
\begin{proof}
    Firstly, an outline is that: 
    \begin{itemize}
        \item Step 1: Solve the characteristic system with initial data $(x_0, u_0, p_0)$ on $\Sigma$. This gives us 
        \[
            (x(s,x_0), u(s,x_0), p(s,x_0)),
        \]
        which we can solve by using ODE theory.
    \end{itemize}
    \item Step 2: Show that the map 
    \[
        \Sigma \times[-\varepsilon, \varepsilon] \ni\left(x_{0}, s\right) \mapsto x\left(x_{0}, s\right) \in \mathbb{R}^{n}
    \]
    is a local diffeomorphism with inverse 
    \[
        x\to (x_0, s).
    \]
    \item Step 3: Define 
    \[
        u(x(s,x_0)) = (s,x_0).
    \]
    This is true if a solution $u$ exists. 

    The main difficulty is that at the end of our construction, we get the function 
    \[
        z\left(s, x_{0}\right)=u(x), \quad x=x\left(s, x_{0}\right), \quad p_{j}\left(s, x_{0}\right) \stackrel{?}{=} \partial_{j} z(x).
    \]
    Our final goal is to prove that $p_j(s,x_0) = \partial_j z(s,x_0).$ By construction of our initial data ,we know this is true at $s=0$. Ideally, we might want to show $\frac \partial {\partial s}(p_j - \partial_j z)=0$. Instead, we will have a weaker version that works: 
    \[
        \frac{\partial}{\partial s}\left(p_{j}-\partial_{j} z\right)=\operatorname{coeff} \cdot \left(p_{j}-\partial_{j} z\right)
    \]
    which is a linear ODE for $p_j - \partial_j z$. 

    Our preliminary step is to show that $F(x,z,p)=0$. This is true on $\Sigma$, i.e., when $s=0$. Compute 
    \[
        \frac{d}{ds}F(x,z,p) = F_x \cdot \dot x + F_z \cdot \dot z + F_p \cdot \dot p = 0.
    \]
    Next, compute $\frac{\partial}{ \partial s}(p_j - \partial_j z)$. We have 
    \[
        \frac{\partial p_j}{\partial s} = -F_{x_j}  - F_z\cdot p_j,
    \]
    but to calculate $\frac{\partial}{\partial s} \partial_j z$, we need to use $\dot z = F_p \cdot p$. $\frac{\partial}{\partial s} = \sum_k F_{p_k} \cdot \frac{\partial}{\partial x_k}$, where $F_{p_k}$ has variable coefficients. So the derivatives do onot commute. We can explicitly compute 
    \[
        \frac{d}{ds}F(x,z,p) = \sum_k  \sum_k F_{p_k} \partial_k \partial_j z.
    \]
    Since 
    \[
        \partial_{j} \dot{z}=\partial_{j}\left(\sum_k F_{p_{k}} \partial_{k} z\right)=\sum_k F_{p_{k}} \partial_{j} \partial_{k} z+\sum_k \partial_{j} F_{p_{k}},\cdot \partial_{k} z
    \]
    which gives 
    \[
        \frac{\partial}{\partial s} \partial_{j} z=\partial_{j} \dot{z}-\sum_k \partial_{j} F_{p_{k}} \cdot \partial_{k} z.
    \]
    So we get 
    \[
        \begin{aligned}
            \frac{\partial}{\partial s}\left(p_{j}-\partial_{j} z\right) &=-F_{x_{j}}-F_{z} \cdot p_{j}-\partial_{j} \dot{z}+\sum_k\partial_{j}\left(F_{p_{k}}\right) \cdot \partial_{k} z \\
            &=-F_{x_{j}}-F_{z} \cdot p_{j}-\sum_k\partial_{j}\left(F_{p_{k}} \cdot p_{k}\right)+\sum_k\partial_j\left(F_{p_{k}}\right) \partial_{k} z \\
            &=-F_{x_{j}}-F_{z} \cdot p_{j}-\sum_k F_{p_{k}} \partial_{j} p_{k} \\ & \underbrace{- \sum_k p_{k}\left(F_{x_{j} p_{k}}+F_{z p_{k}} \partial_{j} z+F_{p_{\ell} p_{k}} \partial_{j} p_{\ell}\right)+ \sum_k \partial_{k} z(F_{x_{j} p_{k}}+F_{z p_{k}} \partial_{j} z+F_{p_{\ell} p_{k}} \partial_{j} p_{\ell})}_{\sum_k -\left(p_{k}-\partial_{k} z\right) \cdot \partial_{j} F_{p_{k}}} \\
            &=-F_{x_{j}}-F_{z} \cdot p_{j}-\sum_kF_{p_{k}} \partial_{j} p_{k}+\sum_k -\left(p_{k}-\partial_{k} z\right) \cdot \partial_{j} F_{p_{k}}
            \end{aligned}
    \]
    We also have
    \[
        F_{x_{j}}+F_{z} \cdot \partial_{j} z+\sum_k F_{p_{k}} \partial_{j} p_{k}=0
    \]
    by taking $\frac{\partial}{\partial x_j}$ of our earlier computation. This last term $F_{p_k} \cdot \partial_j p_k$ is the same term in the above expression. If we substitute, we get
    \[
        \frac{\partial}{\partial_{s}}\left(p_{j}-\partial_{j} z\right)=-F_{z}\cdot \left(p_{j}-\partial_{j} z\right)-\sum_k\partial_{j} F_{p_{k}}\cdot \left(p_{k}-\partial_{k} z\right),
    \]
    which is a linear system.
    Therefore, $z$ is the solution to our equation, and we are done. \qed 
\end{proof}

\subsection{Problems in standard form}
\begin{example}
    Begin with the equation
    \[
        u_{t}+F(t, x, u, \partial u)=0
    \]
    We will label $u_t$ as $\tau$, $u$ as $z$ and $\partial u$ as $p$. So we get the equation: 
    \[
        \widetilde{F}(t, x, z, \tau, p)=\tau+F(t, x, z, p)=0,
    \]
    and the system
    \[
    \begin{cases}
        \dot t = 1 (\text{so } s = t) \\ 
        \dot x = F_p \\
        \dot z = F_p\cdot p + \tau = F_p \cdot p - F \\ 
        \dot p = -F_x -  - F_z \cdot p \\
        \dot \tau = -F_t - F_z \cdot \tau 
    \end{cases}
    \]
    In the midlle 3 euqations, we have no $\tau$ terms, so we can discard the last equation. Another way to think of this is that $\widetilde{ F} =0$, so $\tau$ is already given as $-F$. So we get smaller system 
    \[
    \begin{cases}
        \dot x = F_p \\
        \dot z = F_p\cdot p - F\\ 
        \dot p = -F_x  -F_z \cdot p
    \end{cases}
    \]
    The price we pay is the extra $F$ term in the second equation, compared to before.
\end{example}

\begin{remark}
    Solutions are local, near $\Sigma$, until characteristics may intersect. There is no way to continue solutions in general past this intersection of characteristics. For specific classes of problems, however, there is hope. 
\end{remark}

\begin{example}
    [Hamiton flow]
    Suppose we have an equation 
    \[
        H(x, \partial u) + u_t = 0
    \]
     which does not depend directly on $u$. Then we get 
    \[
    \begin{cases}
        \dot x = H_p \\
        \dot p = -H_x \\
        \dot z  = H_p\cdot p - H
    \end{cases}
    \]
    The first two equations do not depend on $z$, so we can discard the last equation, solve the first two questions first, and integrate the last equation at the end. 

    This type of system is called a \textbf{Hamilton Flow}. Hamilton flows play a role in symplectic geometry. Many PDEs can be interpreted as Hamiltonian flows. The Hamilton-Jacobi euqations are of the form 
    \[
        u_t + H(x,\partial u) = 0.
    \]
\end{example}

Next time, we will do a bit of variational calculus to not only solve Hamilton-Jacobi equations but to also see how we may extend a solution past a point where characteristics intersect. In a Hamilton flow, the characteristics only depend on $(x,p)$. When characteristics intersect, they may have the same $x$ but different $p=\partial u$. We will try to continue the solution in a way such that $\partial u$ has a jump discontinuity.

\begin{example}
    Consider the equation 
    \[
    \begin{cases}
        u_t + \frac 1 2 | \partial_x u|^2 = 0 \\
        u(0) = u_0.
    \end{cases}
    \]
    Here, $H(p) = \frac 1 2 p^2$, and we get the system 
    \[
    \begin{cases}
        \dot x = p\\
        \dot p = 0
    \end{cases}
    \]
    Here, the characteristics are straight lines, with $p(0)=\partial_x u_0$.
\end{example}

\begin{example}
    [Eikonal equation] The equation 
    \[
        |u_t|^2 -|\partial_x u|^2 = 0.
    \]
    is not int the form we have talked about already. This gives 
    \[
        u_t = \pm |\partial_x u|,
    \]
    so we will get 2 solutions.
\end{example}
