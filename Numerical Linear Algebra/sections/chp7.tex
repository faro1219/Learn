\newpage
\section{Lecture 7: Von-Neumann Stability Analysis} 
\begin{itemize}
    \item Plan of today: Von-Neumann stability analysis (theory)
    \item Goal: General method to compute $\|B\|_{2,h}, \|B^n\|_{2,h}$ for a finite difference operator $B$ acting on $\ell _h^2$.
\end{itemize}
\vspace{1em}
\hrule


%────────────────────────────────────────
\begin{definition}
[Z-transform]
\label{def: Z-transform}
Z-transform is a bijection $Z$ from $\ell^2(Z)$ to $\L^2(-\pi,\pi)$: 
\[
    u \mapsto \hat u (\xi ) = \sum_{j=-\infty}^\infty u_j e^{-ij \xi }.  
\]
The inverse map $Z^{-1}$ is given by: 
\[
    f(\xi ) \mapsto f_j^{\vee} = \frac{1}{2\pi } \int_{-\pi }^{\pi} f(\xi ) e^{ij \xi}\; d\xi. 
\]
\end{definition}
%────────────────────────────────────────


%────────────────────────────────────────
\begin{proposition}
[Parseval]
\label{prop: Parseval}
The Parseval equality for Z-transform is: $\sum_j |u_j|^2 = \frac{1}{2\pi } \int_{-\pi}^\pi |\hat u (\xi )|^2 \, d \xi.$ 
\end{proposition}
%────────────────────────────────────────
This is equivalent to say $\|u\|_{\ell^2} = \frac{1}{\sqrt{2\pi } } \|\hat u\|_{L^2}$. Since $\|u\|_{\cB_h} = \sqrt{h \sum_{j}|u_j|^2} = \sqrt{h}\|u\|_{\ell ^2} = \sqrt{\frac{h}{2\pi }}  \|\hat u \|_{L^2}$, we get $\|u\|_{\cB_h} = \|\sqrt{\frac{h}{2\pi } }Zu\|_{L^2}$. Hence, $\sqrt{\frac{h}{2\pi }} Z$ is unitary from $\cB_h$ to $L^2(-\pi ,\pi )$.  

Now consider the shift operator $S{u_j} = u_{j+1}$ on $\ell_h^2$. The matrix presentation is: 
\[
S = \begin{pNiceMatrix}[xdots/line-style=solid]
    0  & 1 &   & 0      \\
          & \Ddots &\Ddots   & \\
     &  & & 1 \\
    0      & &  & 0  \\
   \end{pNiceMatrix}. 
\]
The inverse of $S$ shifts right by:$S^{-1}u_j = u_{j-1}$.  

A general finite difference operator (or a banded Toeplitz matrix) has the form: 
\[
    \cB = \sum_{m=m_1}^{m_2} c_m S^m, \quad B_{ij} = c_{j-i}. 
\]


%────────────────────────────────────────
\begin{example}
\label{eg: postive derivative operator}
\begin{itemize}
    \item []
    \item Positive gradient: $ D_x^+ u = \frac{S-I}{h}.   $ 
    \item Heat equation scheme: $\cB = \nu S^{-1} + (1-2v) S^0 + \nu S^1$. 
\end{itemize}
\end{example}
%────────────────────────────────────────


%────────────────────────────────────────
\begin{note}
In numerical linear algebra, the index convention for Toeplitz operators is often reversed: $B_{ij} = c_{i-j}$. The reason is that it's better to see it's a convolution in numerical linear algebra. 
\[
    \cB u_j = \sum_{m} c_{-m} u_{j+m}   = \sum_{\ell } u_{j-\ell } c_{\ell }. 
\]
\end{note}
%────────────────────────────────────────


%────────────────────────────────────────
\begin{theorem}
[Z-transform diagonalize operator]
\label{thm: Z-transform diagonalizeoperator}
The Z-transform diagonalizes any finite difference operator $\cB = \sum_m c_m S^m$. 
\end{theorem}
%────────────────────────────────────────
%────────────────────────────────────────
\begin{proof}
We are going to define $\cG$ and show the following diagram:
\[\begin{tikzcd}[ampersand replacement=\&]
	{\ell_h^2} \&\& {\ell_h^2} \\
	\\
	{L^2(-\pi,\pi)} \&\& {L^2(-\pi,\pi)}
	\arrow["Z"', from=1-1, to=3-1]
	\arrow["Z", from=1-3, to=3-3]
	\arrow["\cG"', from=3-1, to=3-3]
	\arrow["\cB", from=1-1, to=1-3]
\end{tikzcd}\]
In fact, this is equal to say $ZB = \cG Z$. Note that $\cG f(\xi ) =G(\xi ) f(\xi)$. Hence, $\cG$ is diagonal.  
\begin{align*}
    (z\cB)[u](\xi) = \hat {\cB u} (\xi ) &= \sum_j \cB u_j e^{-ij\xi}
    = \sum_j(\sum_m c_m u_{j+m}) e^{-ij\xi}\\ 
    &= \sum_m \sum_j c_m u_{j+m} e^{-ij\xi} 
     = \sum_m \sum_{\ell} c_m u_\ell  e^{-j(\ell -m)\xi} \\ 
    & = \underbrace{(\sum_m c_m e^{im\xi})}_{G(\xi)} \underbrace{(\sum_\ell  u_l e^{-il \xi})}_{\hat u (\xi)} 
     = (\cG \hat u ) (\xi ) = (\cG Z)[u] (\xi ). 
\end{align*}
\end{proof}
%────────────────────────────────────────

Here $\cG(\xi )$ is known as the amplification factor: 
\[
    \cB{u_j} = \sum_m c_m u_{j+m} \mapsto G(\xi) = \sum_m c_m e^{im\xi}. 
\]
We can think of $w_j = e^{ij \xi }$ as an infinite column vector indexed by $j$ with $\xi $ fixed. Hence, 
\[
    \cB w_j = \sum_{m} c_m w_{j+m} = \sum_{m} c_m e^{i(j+m)\xi } = \left( \sum_{m} c_m e^{im\xi}  \right) e^{ij \xi } = \left( \sum_{m} c_m e^{im\xi}  \right)w_j. 
\]
Therefore, $ \cB w = G(\xi ) w $ is like an eigenvalue decomposition.  This is true in $\ell ^\infty$ since $w\in \ell ^\infty$ but no in $\ell ^2$. $w$ is not square summable.  In fact, the operator $\cB$ (on $\ell_h^2$) doesn't have any eigenvalues since none of the candidate eigenvectors are in the space.  


%────────────────────────────────────────
\begin{definition}
[Spectrum of operator]
\label{def: Spectrum of operator}
Given an operator $\cB$, the spectral is: 
\[
    \sigma (\cB) \coloneqq \{\lambda \in \CC: \lambda I - \cB \text{ is not invertible }\} 
\]
\end{definition}
%────────────────────────────────────────


%────────────────────────────────────────
\begin{proposition}
[Spectrum of $\cB$]
\label{prop: Spectrum of cB}
Given $\cB = \sum_{m} c_m S^m$, 
\[
    \sigma(\cB) = \cG([-\pi, \pi]). 
\] 
\end{proposition}
%────────────────────────────────────────

Nevertheless, since $Z$ is unitary, we have diagonalized $\cB$ and we will see next time that: 
\[
    \|cB\|_{\ell_h^2} = \|\cG\|_{L^2(-\pi, \pi)} = \|G\|_{L^\infty (-\pi,\pi)}. 
\]
and 
\[
    \|\cB\|_{ \ell_h^2 }  = \| \cG^n\|_{ L^2 } =  \| G^n\|_{ L^\infty }.  
\]
