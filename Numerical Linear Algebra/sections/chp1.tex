\newpage
\chapter{Finite Difference and Spectral Methods for Parabolic PDEs}
\section{Lecture 1: Introduction to Heat Equation}

\subsection{Introduction to PDEs} 
Examples of PDEs: 
\begin{itemize}
    \item Linear PDEs: 
    \begin{itemize}
        \item Wave equation: $ u_{tt} = c^2 \Delta u $, $ \Delta = \nabla^2 $ is the Laplacian 
        \item Transport equation: $ u_t + a u_x =0 $, solution is $ u(x,t)=u_0(x-at) $ 
    \end{itemize}
    \item Nonlinear PDEs: inversed Burger's equation: $ u_t + uu_x =0 $, speed depends on the height  
\end{itemize}

In detail, we have 3 classes of second-order linear PDEs: 
\begin{itemize}
    \item Parabolic PDEs: 
    \begin{itemize}
        \item Heat equation: $ u_t = \alpha  \Delta  u $ 
        \item Schrodinger equation: $ i\hbar u_t = (-\frac{\hbar ^2}{2m}\Delta + V(x,t))u $. $\hbar$ the reduced Planck constant, $m$ the mass, $V$ the potential. 
    \end{itemize}
    \item Elliptic PDEs: 
    \begin{itemize}
        \item Poisson: $ -\Delta u =f $ 
        \item Linear elasticity: $ \mu \Delta u + (\lambda +\mu ) \nabla (\nabla \cdot u) = f $ 
        \item Stokes: \[
        \begin{cases}
            -\mu \Delta u + \nabla p = f \\ 
            \nabla \cdot u = 0
        \end{cases} 
        \] 
    \end{itemize}
    \item Hyperbolic PDEs: 
    \begin{itemize}
        \item Elastic vibrations: $ \rho u_{tt} = \mu \Delta u + (\lambda + \mu ) \nabla(\nabla\cdot u) $ 
        \item Maxwell: 
        \[
            \begin{cases}
                \nabla \cdot E = 4\pi \rho,  \nabla \cdot B = 0 \\ 
                \nabla \times E = -\frac{1}{c}\frac{\partial B}{\partial t}, \nabla \times B = \frac{1}{C}\left( 4\pi J + \frac{\partial E}{\partial t} \right) 
            \end{cases}
        \]
    \end{itemize}
    \item Nonlinear examples: 
    \begin{itemize}
        \item Navier-Stokes: 
        \[
            \begin{cases}
                \rho (ut + u \Delta u) = -\nabla p + \mu \Delta u \\ 
                \Delta u = 0. 
            \end{cases}
        \]
        \item Eikonal equation: $ |\nabla u| = 1 $, first arrival time of a signal
        \item Viscons Burger's: $ u_t + uu_x = \nu u_{xx} $, a 1d variant of Navier Stokes 
        \item KdV: $ u_t + uu_x = -\nu^2 u_{xxx} $
        \item Traffic equation: $ \rho_t + (\rho U(\rho))_x = 0 $    
    \end{itemize}
\end{itemize}

\begin{note}
    \begin{itemize}
        \item []
        \item Unlike ODEs, there's no general theory of PDEs. 
        \item Each type of equation has special features that must be understood and incorporated into a numerical method. (e.g. if the solution forms shocks, the numerical method must be able to handle discontinuities) 
        \item Boundary conditions are often a major challenge for PDEs  in complex geometry. 
    \end{itemize}
\end{note}

\subsection{Heat equation: Background} 
The non-dimensional, 1-d variant heat equation is: 
\[
    u_t = u_{xx}. 
\]
Here we  have two different setups: 
\begin{itemize}
    \item Rod of finite length: $0\le x\le L$ 
    \item Infinite domain: $ -\infty<x < \infty $ 
\end{itemize}

For the first setup, one example is: 
\[
    \begin{cases}
        u_t - u_xx = 0 (\text{ or } f) \\ 
        u(x,0) = g(x) \\ 
        u(0) = u(\pi) = 0. 
    \end{cases}
\]
Here $g$ is the initial temperature distribution and $f$ is the source of heat. We can assume there are some candles under an iron stick. The goal is to find $u(x,t)$ for $t>0,0\le x\le \pi $.  

Consider the separation of variables: $u(x,t) = X(t) T(t)$, we got 
\[
    \frac{T_t}{T}= \frac{X_{xx}}{X}=\lambda. 
\]
Impose the b.c.'s $X(0) = X(\pi) = 0$, we get solutions: 
\[
    \begin{cases}
        X(x) = \sin kx, \lambda_k = -k^2 \\ 
        T(t) = e^{-k^2t}
    \end{cases}
\]
where $k=1,2,3,\ldots $.  


%────────────────────────────────────────
\begin{definition}
[superposition]
\label{def: superposition}
Use a Fourier series to represent the initial condition.  
\[
    g(x) = \sum_{k=1}^{\infty} c_k \sin k x, \quad c_k = \frac{2}{\pi }\int_0^\pi  g(x) \sin k x\, dx.   
\]
\end{definition}
%────────────────────────────────────────

Now we evolve each component of $g(x)$ independently: 
\[
    u(x,t) = \sum_{k=1}^{\infty} c_k e^{-k^2t} \sin kx.  
\]


%────────────────────────────────────────
\begin{note}
For the backward heat equation $ (u_t = -u_{xx}) $, the Fourier modes grow exponentially in time (rather than decaying).  
\end{note}
%────────────────────────────────────────


%────────────────────────────────────────
\begin{example}
\label{eg: he}
Assume $g(x) = x(\pi-x), x\in[0,\pi]$, then 
\[
    c_k = \frac{2}{\pi } \int_{0}^{\pi } g(x) \sin kx \, dx = 
    \begin{cases}
        0, \quad &k \text{ even }\\ 
        \frac{8}{\pi k^3}, \quad &k \text{ odd }
    \end{cases} 
\]
Hence with $u_t = -u_{xx}$, we have $ u(x,t) = \sum_{k \text{ odd }} \frac{8}{\pi k^3} e^{k^2 t} \sin kt   $. So the formula for $u$ diverges for any $t>0$. 
\end{example}
%────────────────────────────────────────
