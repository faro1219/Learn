\newpage 
\section{Lecture 4: Functional Analysis of Finite Difference}
Recap: 
\begin{itemize}
    \item Truncation error for $D_t^{+} u=D_x^{+} D_x^{-} u$
    \item Statement of Lax-Richtmyer equivalence theorem 
\end{itemize}
Topics today: 
\begin{itemize}
    \item Functional analysis review 
    \item Proof of Lax-Richtmyer 
\end{itemize}

\vspace{1em}
\hrule 
\vspace{1em} 

\subsection{Functional analysis} 
Functional analysis is the study of linear algebra in infinite dimensions.  

%────────────────────────────────────────
\begin{definition}
[Banach space]
\label{def: Banach space}
A complete normed vector space.
\end{definition}
%────────────────────────────────────────

%────────────────────────────────────────
\begin{definition}
[Hilbert space]
\label{def: Hilbert space}
A complete inner product space.
\end{definition}
%────────────────────────────────────────
We skip the introduction to norms, inner products and linear functionals. 

\subsection{Proof of Lax-Richtmyer}

Note that Thm~\ref{thm: Lax-Richtmyer equation} is equivalent to say 
\[
    \text{ consistency } + \text{ stability } \Rightarrow \text{ convergence }. 
\]

\begin{proof}[Proof of Thm~\ref{thm: Lax-Richtmyer equation}] 
We denote errors by: 
\[
    e_j^n = u_j^n - u(jh, nk). 
\]
Note that we have 
\[
    u_j^{n+1}=B(k) u^n_j, \quad u(jh, (n+1)k) = B(k) u(jh, nk) + k \tau_j^n. 
\]
We subtract them and will get 
\[
    e_j^{n+1} = B(k) e_j^n - k \tau_j^n. 
\]
By induction, we will get 
\[
    e_j^n = B(k)^n e_j^0 - B(k)^{n-1} k \tau_j^0 - B(k)^{n-2} k \tau_j^1 - \cdots - B(k) k\tau_j^{n-2} - k \tau_j^{n-1}. 
\]
Take norms and use triangle inequality and $\|B(k)^l\|\le K$, we get 
$$
\left\|e^n\right\| \leqslant K \underbrace{\left\|e^0\right\|}_0+K k\left(\left\|\tau^0\right\|+\left\|\tau^{\prime}\right\|+\cdots+\left\|\tau^{n-1}\right\|\right) \le K n k \max _l\left\|\tau^l\right\|, n k \leqslant T
$$
In our case, we have show that 
\[
    \|\tau^l\|_\infty \le \begin{cases}
        O(k+h^2) \quad & \nu \neq \frac{1}{6}\\ 
        O(k^2 +h^4) & \nu =\frac{1}{6}
    \end{cases}
\]
we will show that if $\nu\le \frac{1}{2}$, $K=1$ and $\epsilon =\infty$.  Hence, we have 
\[
    \|e^n\|_{l^\infty} \le \begin{cases}
        TM\left( \frac{k}{2} + \frac{h^2}{12} \right) \quad & \nu \neq \frac{1}{6}, 0<\nu \le \frac{1}{2} \\ 
        TM\left( \frac{k^2}{6} + \frac{h^2}{360} \right) = \frac{TMh^4}{135} &\nu = \frac{1}{6}
    \end{cases}
\]
This is true for all $n$ satisfying $0\le nk\le T$.  So the maximal value of the error on the grid goes to zero as $k,h\to 0$ with $\nu = \frac{k}{h^2}$ held fixed. 
\end{proof}